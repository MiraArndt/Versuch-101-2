\section{Auswertung}
\label{sec:Auswertung}

\subsection{Errechnete Theoriewerte}

Mit Hilfe der gemessenen Daten und Gleichungen \ref{eq:zyl1} und \ref{eq:zyl2}
lassen sich die Trägheitsmomente der beiden Zylinder
als

\begin{equation*}
  I=0,787 \cdot 10^{-3}\,\si[inter-unit-product=\cdot]{\kilo\gram\meter\squared}
\end{equation*}

\noindent und als

\begin{equation*}
  I=15,432 \cdot 10^{-3}\,\si[inter-unit-product=\cdot]{\kilo\gram\meter\squared}
\end{equation*}

\noindent berechnen. Bei der Puppe wird zusätzlich 
Gleichung \ref{eq:Kugel} und der Satz von Steiner \ref{eq:steiner}
benötigt. Für die erste Position ergibt sich ein
Trägheitsmoment von

\begin{equation*}
  I=7,916 \cdot 10^{-5}\,\si[inter-unit-product=\cdot]{\kilo\gram\meter\squared}
\end{equation*}

\noindent und für die zweite ein Trägheitsmoment von

\begin{equation*}
  I=1,591 \cdot 10^{-5}\,\si[inter-unit-product=\cdot]{\kilo\gram\meter\squared}
\end{equation*}

\noindent Dabei wurde die Dichte des Holzes der Puppe
als $670\,\si{\kilo\gram \per \cubic \meter}$ (siehe \cite{holz})
angenommen.
