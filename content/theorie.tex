\section{Theorie}
\label{sec:Theorie}

Das Trägheitsmoment $I$ wird zusammen mit dem
Drehmoment $\vec{M}$ und der Winkelbeschleunigung $\dot{\omega}$
benötigt, um Rotationsbewegungen vollstängig zu charakterisieren.
Eine Punktförmige Masse $m$ hat das Trägheitsmomen

\begin{equation}
    I=m \cdot r^2,
\end{equation}

\noindent wobei $r$ den Abstand zur Drehachse angibt.
Bei einem ausgedehnten Körper mit dikreten Massenelementen 
werden alle Massenelemente $m$ 
mit dem entsprechenden Abstand $r$ zusammenaddiert und 
ergeben somit das Trägheitsmoment

\begin{equation}
    I=\sum_i r_i^2 \cdot m_i.
    \label{eq:a}
\end{equation}

\noindent Beim Übergang zum kontinuierlichen System wird 
Ausdruck \ref{eq:a} zu

\begin{equation}
    I=\int r^2 \, \symup{d}m.
\end{equation}

\noindent Um das Berechnen des Trägheitsmoments zu vereinfachen
gibt es Hilfsmittel wie den Satz von Steiner, welcher
einen Zusammenhang 

\begin{equation}
    I=I_S+m \cdot a^2
    \label{eq:steiner}
\end{equation}

\noindent zwischen dem Trägheitsmoment bezogen auf eine
Drehachse $I_S$ durch den Schwerpunkt und einem Trägheitsmoment
zur Drehachse parallel dazu $I$ herstellt. Dabei bezeichnet $a$
den Abstand der beiden Drehachsen.

Ein komplexer Körper kann zudem in einfache Teile aufgeteilt werden, um
so die Trägheitsmomente von möglichst symetrischen Körpern berechenen
zu können. Diese lassen sich besonders einfach berechnen, oder
sind sogar in Tabellen aufgeführt und die 
entsprechenden Trägheitsmomente können schließlich wieder addiert werden.

Das Drehmoment $\vec{M}$ berechnet sich als

\begin{equation}
    \vec{M}=\vec{F} \symup{\times} \vec{r},
    \label{eq:fa1}
\end{equation}

\noindent wobei $\vec{F}$ die Kaft auf einen
Rotationskörper und $\vec{r}$ den Abstand des 
Angreifpunktes der Kraft zur Drehachse angibt.

Wirkt bei einem schwingungsfähigem Rotationssystem
mit Trägheitsmoment $I$
eine Feder mit Winkelrichtgröße $D$ der Drehung entgegen,
so gilt für die Schwingungsdauer $T$ bei kleinen 
Auslenkungswinkeln $\phi$

\begin{equation}
    T=2\pi \sqrt{\frac{I}{D}}.
    \label{eq:fa2}
\end{equation}

\noindent Für kleine Winkel $\phi$ gilt außerdem der
harmonische Zusammenhang

\begin{equation}
    M=D \cdot \phi.
    \label{eq:fa3}
\end{equation}

\noindent Für die untersuchten Körper werden 
die, zur berechnung der Trägheitsmomente benötigten,
Formeln dem Skript der Physik I \cite{Skript} entnommen.

Das Trägheitsmoment eines Vollzylinders mit Radius $R$
und Masse $m$, der sich um seine
Symmetrieachse dreht berechnet sich nach

\begin{equation}
    I=\frac{1}{2} \cdot m \cdot R^2.
    \label{eq:zyl1}
\end{equation}

\noindent Dreht sich ein Vollzylinder um eine Achse senkrecht
zu seiner Symmetrieachse durch den Schwerpunkt, so wird das Trägheitsmoment durch

\begin{equation}
    I=\frac{1}{4} \cdot m \cdot R^2 + \frac{1}{12} \cdot m \cdot h^2
    \label{eq:zyl2}
\end{equation}

\noindent beschrieben. Für das Trägheitsmoment einer
Vollkugel  mit Masse $m$ und Radius $R$ gilt

\begin{equation}
    I=\frac{2}{5} \cdot m \cdot R^2.
    \label{eq:Kugel}
\end{equation}